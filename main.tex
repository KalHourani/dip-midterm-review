\documentclass[a4paper]{article}

%% Language and font encodings
\usepackage[english]{babel}
\usepackage{amsthm}
\usepackage[utf8x]{inputenc}
\usepackage[T1]{fontenc}
\usepackage{listings}
\usepackage{algpseudocode, algorithm}
\usepackage{mathtools}
\usepackage{enumerate}
\usepackage{tabu}
\usepackage{xcolor}
\usepackage{tikz, tikz-qtree}
\usepackage{graphviz}
\usepackage{subfig}

%% Sets page size and margins
\usepackage[a4paper,top=3cm,bottom=2cm,left=3cm,right=3cm,marginparwidth=1.75cm]{geometry}

%% Useful packages
\usepackage{amsmath, amssymb}
\usepackage{graphicx}
\usepackage[colorinlistoftodos]{todonotes}
\usepackage[colorlinks=true, allcolors=blue]{hyperref}

\newenvironment{solution}{\begin{proof}[\textnormal{\textbf{Solution}}]}{\end{proof}}
\newenvironment{exercise}[1]{\begin{proof}[\textnormal{\textbf{Exercise #1:}}]\phantom{\qedhere}}{\end{proof}}
\newenvironment{lemma}{\begin{proof}[\textnormal{\textbf{Lemma}}]\phantom{\qedhere}}{\end{proof}}

\renewcommand\thesubfigure{\arabic{subfigure}}

\begin{document}
\begin{titlepage}\pagenumbering{gobble}
	\centering
	{\scshape\LARGE University of Houston\par}
	\vspace{1cm}
	{\scshape\Large Midterm Review \par}
	\vspace{1.5cm}
	{\huge\bfseries COSC  4393\par}
	{\huge\bfseries Digital Image Processing\par}
	\vspace{0.5cm}
	{\large\bfseries Praniv Mantini\par}
	\vspace{2cm}
	{\Large\itshape Khalid Hourani\par}
	\vspace{0.5cm}
	{\large \par} March 7, 2019
	\vfill

% Bottom of the page
\end{titlepage}
\vspace*{\fill}\begin{center}{\Huge This page intentionally left blank.}\end{center}\vspace*{\fill}\thispagestyle{empty}\clearpage
\pagenumbering{arabic}

For an $M\times N$ image $I$, the \textbf{Discrete Fourier Transform} at a pixel $u, v$ is given by 

\begin{align*}
 \tilde{I}(u, v)=\sum_{i=0}^{M-1}\sum_{j=0}^{N-1}I(i, j)e^{-2\pi\sqrt{-1}\left(\frac{ui}{M}+\frac{vj}{N}\right)}
\end{align*}

For a square image, we have $N=M$ and 

\[ \tilde{I}(u, v)=\sum_{i=0}^{N-1}\sum_{j=0}^{N-1}I(i, j)e^{-\sqrt{-1}\frac{2\pi}{N}\left(ui+vj\right)}\]

A complex-valued function $f$ is \textbf{Conjugate Symmetric} or \textbf{Hermitian} if $f^*(z)=f(-z)$. In the case of the DFT, we have 
\begin{align*}
 \tilde{I}(-u, -v)&=\sum_{i=0}^{N-1}\sum_{j=0}^{N-1}I(i, j)e^{-\sqrt{-1}\frac{2\pi}{N}\left(-ui+-vj\right)}
		\\&=\sum_{i=0}^{N-1}\sum_{j=0}^{N-1}I(i, j)\left(\cos{\left(-\frac{2\pi}{N}\left(-ui+-vj\right)\right)}+\sqrt{-1}\sin{\left(-\frac{2\pi}{N}\left(-ui+-vj\right)\right)}\right)
		\\&=\sum_{i=0}^{N-1}\sum_{j=0}^{N-1}I(i, j)\left(\cos{\left(-\frac{2\pi}{N}\left(ui+vj\right)\right)}-\sqrt{-1}\sin{\left(-\frac{2\pi}{N}\left(-ui+-vj\right)\right)}\right)
		\\&=\tilde{I}^*(u, v)
\end{align*}

Equivalently, a complex valued function is Conjugate Symmetric if and only if its real part is even and its imaginary part is odd. The real part of the DFT is just \[\sum_{i=0}^{N-1}\sum_{j=0}^{N-1}I(i, j)\cos{\left(-\sqrt{-1}\frac{2\pi}{N}\left(ui+vj\right)\right)}\] and the imaginary part is \[\sum_{i=0}^{N-1}\sum_{j=0}^{N-1}I(i, j)\sin{\left(-\sqrt{-1}\frac{2\pi}{N}\left(ui+vj\right)\right)}\] which are clearly even and odd, respectively.

The \textbf{magnitude} of a complex number $z=x+iy$ is given by $|z|=x^2+y^2$. To see that the magnitude of the DFT is symmetric, evaluate 

\begin{align*}
 \left|\tilde{I}(u, v)\right|=\left|\sum_{i=0}^{N-1}\sum_{j=0}^{N-1}I(i, j)e^{-\sqrt{-1}\frac{2\pi}{N}\left(ui+vj\right)}\right|
\end{align*}


For example, to compute the DFT of the following matrix: \[\begin{bmatrix}5&7\\8&3\end{bmatrix}\]

We evaluate \begin{align*}
	     \tilde{I}(0,0)&=\sum_{i=0}^1\sum_{j=0}^{1}I(i, j)e^{-\sqrt{-1}\frac{2\pi}{2}\left(0i + 0j\right)}
			 \\&=\sum_{i=0}^1\sum_{j=0}^{1}I(i,j)
			 \\&=I(0,0)+I(0,1)+I(1, 0)+I(1,1)
			 \\&=5+7+8+3
			 \\&=23
	    \end{align*}
	    
	    \begin{align*}	     
	    \tilde{I}(0,1)&=\sum_{i=0}^1\sum_{j=0}^1I(i, j)e^{-\sqrt{-1}\frac{2\pi}{2}\left(0i + j\right)}
			\\&=\sum_{i=0}^1\sum_{j=0}^1I(i,j)e^{-\sqrt{-1}\pi j}
			\\&=\sum_{i=0}^1\sum_{j=0}^1I(i,j)\left(\cos{\left(-\pi j\right)}+\sqrt{-1}\sin{\left(-\pi j\right)}\right)
	    \end{align*}
	    Plugging in the values of $i$ and $j$:
	    \begin{align*}
	    \tilde{I}(0,1)&=I(0,0)\left(\cos{\left(-\pi\cdot0\right)}+\sqrt{-1}\sin{\left(-\pi\cdot0\right)}\right)
			\\&+I(0,1)\left(\cos{\left(-\pi\cdot1\right)}+\sqrt{-1}\sin{\left(-\pi\cdot1\right)}\right)
			\\&+I(1,0)\left(\cos{\left(-\pi\cdot0\right)}+\sqrt{-1}\sin{\left(-\pi\cdot0\right)}\right)
			\\&+I(1,1)\left(\cos{\left(-\pi\cdot1\right)}+\sqrt{-1}\sin{\left(-\pi\cdot1\right)}\right)
			\\&=5\left(1+0i\right)+7\left(-1-0i\right)+8\left(1+0i\right)+3\left(-1-0i\right)
			\\&=5-7+8-3
			\\&=3
	    \end{align*}
	    
	    	    \begin{align*}	     
	    \tilde{I}(1,0)&=\sum_{i=0}^1\sum_{j=0}^1I(i, j)e^{-\sqrt{-1}\frac{2\pi}{2}\left(i + 0j\right)}
			\\&=\sum_{i=0}^1\sum_{j=0}^1I(i,j)e^{-\sqrt{-1}\pi i}
			\\&=\sum_{i=0}^1\sum_{j=0}^1I(i,j)\left(\cos{\left(-\pi i\right)}+\sqrt{-1}\sin{\left(-\pi i\right)}\right)
	    \end{align*}
	    Plugging in the values of $i$ and $j$:
	    \begin{align*}
	    \tilde{I}(1,0)&=I(0,0)\left(\cos{\left(-\pi\cdot0\right)}+\sqrt{-1}\sin{\left(-\pi\cdot0\right)}\right)
			\\&+I(0,1)\left(\cos{\left(-\pi\cdot0\right)}+\sqrt{-1}\sin{\left(-\pi\cdot0\right)}\right)
			\\&+I(1,0)\left(\cos{\left(-\pi\cdot1\right)}+\sqrt{-1}\sin{\left(-\pi\cdot1\right)}\right)
			\\&+I(1,1)\left(\cos{\left(-\pi\cdot1\right)}+\sqrt{-1}\sin{\left(-\pi\cdot1\right)}\right)
			\\&=5\left(1+0i\right)+7\left(1-0i\right)+8\left(-1+0i\right)+3\left(-1-0i\right)
			\\&=5+7-8-3
			\\&=1
	    \end{align*}
	   
Finally:

	    \begin{align*}	     
	    \tilde{I}(1,1)&=\sum_{i=0}^1\sum_{j=0}^1I(i, j)e^{-\sqrt{-1}\frac{2\pi}{2}\left(i + j\right)}
			\\&=\sum_{i=0}^1\sum_{j=0}^1I(i,j)e^{-\sqrt{-1}\pi (i+j)}
			\\&=\sum_{i=0}^1\sum_{j=0}^1I(i,j)\left(\cos{\left(-\pi (i+j)\right)}+\sqrt{-1}\sin{\left(-\pi (i+j)\right)}\right)
	    \end{align*}
	    Plugging in the values of $i$ and $j$:
	    \begin{align*}
	    \tilde{I}(0,1)&=I(0,0)\left(\cos{\left(-\pi\cdot(0+0)\right)}+\sqrt{-1}\sin{\left(-\pi\cdot(0+0)\right)}\right)
			\\&+I(0,1)\left(\cos{\left(-\pi\cdot(0+1)\right)}+\sqrt{-1}\sin{\left(-\pi\cdot(0+1)\right)}\right)
			\\&+I(1,0)\left(\cos{\left(-\pi\cdot(1+0)\right)}+\sqrt{-1}\sin{\left(-\pi\cdot(1+0)\right)}\right)
			\\&+I(1,1)\left(\cos{\left(-\pi\cdot(1+1)\right)}+\sqrt{-1}\sin{\left(-\pi\cdot(1+1)\right)}\right)
			\\&=5\left(1+0i\right)+7\left(-1+0i\right)+8\left(-1+0i\right)+3\left(1+0i\right)
			\\&=5-7-8+3
			\\&=-7
	    \end{align*}
	     
This gives the DFT Matrix \[\begin{bmatrix}21&3\\1&-7\end{bmatrix}\]

\end{document}